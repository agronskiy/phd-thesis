%!TEX root = ../gronskiy_phd_thesis.tex
\chapter*{Abstract}

Dealing with noisy inputs to optimization problems has been one of the central
tasks in the field of inference since its invention. The reason is clear: no
data comes without measurement uncertainty.  \textit{Robustly optimizing} under
such uncertainty is the first cornerstone of this thesis. The second cornerstone
stems from an idea of contemplating uncertain optimization problems~---
combinatorial ones in our case,~--- as \textit{large disordered systems} and
analyzing their optimization behavior using the methods of statistical mechanics.
Questions concerning optimization methods per se~--- e.g. related to a specific
optimization procedure or guarantees thereof,~--- are not considered in this work.

More precisely, we first address robust optimization under uncertainty by means
of an approximation set-based approach to inference. Here, \textit{approximation
set-based} approach refers to a family of methods regularizing the Empirical
Risk Minimization (ERM). One performs it by sampling from a set of near-optimal
solutions instead of returning the ERM optimizer.  We will describe this
approach and design proof-of-concept experiments which support its usability. In
addition, we will address one of the known computational bottlenecks of the
approach by proving and testing theoretical results.

Second, we expand the approximation set-based approach to the field of
algorithmic combinatorial optimization under uncertain input, picking as an
example its application to the  Minimum Spanning Tree (MST) problem. In short,
the aim is to perform approximation by means of optimal stopping of the
algorithm. We will derive a fast computation pipeline which is free of
computational bottlenecks mentioned above; we then carry out experiments
revealing an ability of our approach not only to robustly solve MST problems,
but also to establish a score that ranks various algorithms according to their
expected localization error~--- as in model validation.

Next, we gradually shift our focus to Maximum Entropy inference for
combinatorial optimization problems. Considering them as large disordered
particle systems~\citep{book/MezardM09} driven by energy optimization via Gibbs
distributions, we rigorously study a so-called \textit{Gibbs relaxation} of the
approximation set-based approach. Surprisingly, this study leads to a prominent
mathematical problem~--- asymptotically-precise computing of \textit{free
energy},~--- which we then fully solve for a special class of combinatorial
optimization problems. Thus, in this part of the thesis we fulfill a twofold
task: on the one hand, we study properties of Gibbs relaxation of approximate
robust optimization; the other, we prove a theoretical result, which is of an
independent interest in statistical mechanics.

Last, we drift away from approximate inference and remain in statistical
mechanical setting. Inspired by the theoretical results described in the
previous paragrpah, we ask fundamental questions about statistical mechanics of
combinatorial optimization problems and how their properties define their
solution structure. We pick a combinatorial optimization problem (sparse Minimum
Bisection Problem, sMBP) and compare its asymptotic behavior with the one of
well-known Random Energy Model (REM,~\citealp{derrida81}), leading to some
interesting conjectures about differences in their search complexity.

\chapter*{Zusammenfassung}

Von Anfang an war die Einbeziehung des Rauschens in die Eingabe von
Optimierungsproblemen ein zentrales Problem im Bereich der Inferenz. Der Grund
dafür ist klar: Daten sind inherent mit gewissen Unsicherheiten in Ihrer
Erhebung verbunden. \textit{Robuste Optimierung} unter Berücksichtigung solcher Unsicherheiten
ist der erste Eckpfeiler dieser Dissertation. Der zweite Eckpfeiler beruht auf
der Idee, dass man verrauschte Optimierungsprobleme~--- in unserem Fall
kombinatorische,~--- als \textit{grosse ungeordnete Systeme} betrachtet und
deren Optimierungsverhalten mittels statistischer Mechanik analysiert. Die
Fragen, die sich auf Optimierungsverfahren beziehen (z.B. im Zusammenhang mit
einer spezifischen Optimierungsmethode oder deren Effizienz), sind in dieser Arbeit
nicht berücksichtigt.


Erstens konzentrieren wir uns auf robuste Optimierung unter Unsicherheit mittels
des \textit{auf Approximationsmengen-basierten} (approximation set-based)
Ansatzes für Inferenz. Hier bezieht sich der Begriff des Approximationsmengen
basierten Ansatzes auf eine Familie von Methoden, welche die Empirische
Risiko-Minimierung (Empirical Risk Minimization, ERM) regularisieren. Man macht
es, indem man aus einer Menge von nahezu optimalen Lösungen Stichproben zieht,
statt eine ERM-Lösung als Antwort zurückzugeben. Wir beschreiben diesen Ansatz
und führen Proof-of-Concept-Experimente aus, welche die Anwendbarkeit des
Konzepts zeigen. Ausserdem beschäftigen wir uns mit den bekannten rechnerischen
Engpässen diesen Ansatzes, indem wir theoretische Einsichten anbieten und
testen.

Zweitens, erweitern wir den Approximationsmengen-basierten Ansatz auf dem Gebiet
der kombinatorischen Optimierung gegeben Unsicherheit in der Eingabe, indem wir
als Beispiel das Minimale-Spannbaum-Problem auswählen (Minimum Spanning Tree,
MST). Kurz gesagt besteht unser Ziel darin, eine Approximation durch optimales
Anhalten des Algorithmus zu erzielen. Wir präsentieren eine schnelle
Berechnungspipeline, die von den oben erwähnten Berechnungsenpässen befreit ist;
dazu führen wir Experimente durch, die zeigen dass der oben definierte Ansatz
nicht nur in der Lage ist, MST-Probleme zuverlässig zu lösen, sondern auch einen
Ranking Wert einführt, der verschiedene Algorithmen nach ihrem
Lokalisierungsfehler bewertet~--- wie beim Modellvalidierungsverfahren.

Im Anschluss richten wir den Fokus auf die Maximum-Entropie-Inferenz für
kombinatorische Optimierungsprobleme. Wir betrachten sie als grosse ungeordnete
Partikelsysteme~\citep{book/MezardM09}, die ihr Energieniveau mittels
Gibbs-Verteilungen optimieren. Wir führen eine thoretische Untersuchung der
sogenannten \textit{Gibbs-Relaxation} des Approximationsmengen basierten
Ansatzes aus. Unerwartet, führt diese Untersuchung zu einem berühmten
mathematischen Problem~--- der asymptotisch genauen Berechnung der
\textit{freien Energie},~--- welches wir dann für eine spezielle Klasse von
kombinatorischen Optimierungsproblemen vollständig lösen. Somit liefert dieser
Teil  zwei Beiträge: einerseits untersuchen wir Eigenschaften der
Gibbs-Relaxation für robuste Optimierung; andererseits erreichen wir ein
theoretisches Ergebnis, das eine unabhängige Bedeutung für die statistische
Mechanik hat.

Zuletzt entfernen wir uns von robuster Inferenz und bleiben im
statistisch-mechanischen Bereich. Inspiriert durch die oben erwähnten
theoretischen Ergebnisse, stellen wir fundamentale Fragen zur statistischen
Mechanik kombinatorischer Optimierungsprobleme und wie deren Eigenschaften die
Lösungsstruktur definieren. Wir wählen ein kombinatorisches Optimierungsproblem
(sparse Minimum Bisection Problem, sMBP) und vergleichen sein asymptotisches
Verhalten mit dem des bekannten Random Energy Models (REM,~\citealp{derrida81}),
was zu einigen spannenden Vermutungen über Unterschiede in ihrer Suchkomplexität
führt.

% Umgehen mit dem Rauschen in Angaben für Optimierungsporbleme war seit der
% Erfindung eine von zentralen Angaben im Bereich der Inferenz. Grund dafür ist
% klar: Daten kommen immer mit gewisser Unsicherheit in Messungen. \textit{Robuste
% Optimierung} unter solcher Unsichercheit ist der erste Eckpfeiler dieser
% Dissertation. Der zweite Eckpfeiler stammt aus der Idee, dass man verrauschte
% Optimierungsprobleme~--- in unserem Fall kombinatorische,~--- als \textit{grosse
% ungeordnete Systeme} beachtet und deren Optimierungsverhalten mittels
% statistischer Mechanik analysiert. Die Fragen, die sich auf
% Optimierungsverfahren beziehen (z.B.~im Zusammenhang mit spezifischer
% Optimierungsmethode oder deren Leistung), sind in dieser Arbeit nicht
% berücksichtigt.

% Erstens konzentrieren wir uns auf robuste Optimierung unter Unsicherheit mittels
% des auf Approximationsmengen basierten (approximation set-based) Ansatzes für
% Inferenz. Hier bezieht sich der Begriff \textit{Aproximationsmenge-basierter
% Ansatz} auf eine Familie von Methoden, die die Empirische Risiko-Minimierung
% (Empirical Risk Minimization, ERM) regularisieren. Man macht es, indem man aus
% einer Menge von nahezu optimalen Lösungen samplet, statt ERM-Lösung als die
% Antwort zurückzugeben. Diese einfache Idee ist der Vicinal Risiko-Minimierung
% (Vicinal Risk Minimization,~VRM,~\citealp{chapelle01}) sehr nahe, die ein
% ähnliches Konzept verwendet, das versucht, Verbundsrisiko aller nahezu optimalen
% Lösungen gleichzeitig zu minimieren. Wir werden diesen Ansatz beschreiben und
% Proof-of-Concept-Experimente ausführen, die seine Anwendbarkeit beweisen.
% Ausserdem werden wir uns mit den bekannten rechnerischen Engpässen dieses
% Ansatzes beschäftigen, indem wir theoretische Ergebnisse anbieten und testen.

% Zweitens, erweitern wir den Approximationsmengen-basierten Ansatz auf dem Gebiet
% der algorithmischen kombinatorischen Optimierung unter unsicheren Eingaben,
% indem wir als Beispiel die Anwendung auf das Minimales-Spannbaum-Problem
% (Minimum Spanning Tree, MST) auswählen. Kurz gesagt, besteht unseres Ziel darin,
% eine Approximierung zu erlediegen, indem man das Algorithmus optimal Anhält. Wir
% werden eine schnelle Berechnungspipeline präsentieren, die von den oben
% erwähnten Berechnungsengpässen befreit ist; dazu werden wir dann Experimente
% durchführen, die zeigen, dass der oben definierte Ansatz nicht nur in der Lage
% ist, MST-Probleme zuverlässig zu lösen, sondern auch einen Ranking-Wert einfürt,
% der verschiedene Algorithmen nach ihrem erwarteten Lokalisierungsfehler
% rangiert~--- ähnlich zum Modellvalidierungsverfahren.

% Danach fokussieren wir auf die statistische Mechanik kombinatorischer
% Optimierungsprobleme. Wir betrachten sie als grosse ungeordnete
% Partikelsysteme~\citep{book/MezardM09}, die ihren Energieniveau mittels
% Gibbs-Verteilungen optimieren.  Wir führen eine thoretische Untersuchung
% sogenannter \textit{Gibbs-Relaxation} des Approximationsmengen-basierten
% Ansatzes aus. Unerwartet, führt diese Untersuchung zu einem berühmten
% mathematischen Problem~--- der asymptotisch genauer Berechnung der
% \textit{freien Energie},~--- das wir dann für eine spezielle Klasse von
% kombinatorischen Optimierungsproblemen vollständig lösen. Somit ist der Beitrag
% dieses Teils zweifältig: einerseits untersuchen wir Eigenschaften der
% Gibbs-Relaxation für robuste Optimierung; andererseits erreichen wir ein 
% theoretisches Ergebnis, das eine unabhängige Bedeutung für die statistische
% Mechanik hat.

% Zuletzt entfernen wir uns von robuster Inferenz und bleiben im
% statistisch-mechanischen Bereich. Inspiriert durch die oben erwähnten
% theoretischen Ergebnisse, stellen wir fundamentale Fragen zur statistischen
% Mechanik kombinatorischer Optimierungsprobleme und wie deren thermodynamische
% Eigenschaften die Lösungsstruktur definieren. Wir wählen ein kombinatorisches
% Optimierungsproblem (sparse Minimum Bisection Problem, sMBP) und vergleichen
% sein \textit{thermodynamisches} Verhalten mit dem des berühmten Random Energy
% Models (REM,~\citealp{derrida81}), was zu einigen spannenden Vermutungen über
% Unterschiede in ihrer Suchkomplexität führt.


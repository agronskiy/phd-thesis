%!TEX root = ../gronskiy_phd_thesis.tex
\chapter{Concluding Remarks}
\label{ch:conclusion}

\hfill
\begin{minipage}[t]{.75\textwidth}
\textit{``The measure of greatness in a scientific idea is the extent to which
it stimulates thought and opens up new lines of research.''} \\
  \hrule
  \vspace{.2cm}
  \hfill
  \textsc{---  Paul DIRAC}
\end{minipage}
\\[.5cm]

In this thesis, we addressed the problem of robust approximate optimization in
different settings: general, algorithmic and thermodynamic, and also looked into
the thermodynamic behavior of optimization problem in a more general (i.e. not
related to approximation) sense. Because detailed remarks are given at the end
of each chapter, here we give some very general thoughts about possible
directions of the further research.

\section{Approximate Optimization in General}

It would be interesting to extend the approaches presented in
Chapter~\ref{ch:gen_appch} to the case of more than two instances.
While some straightforward generalizations may exist~--- e.g. mechanistically
extend the formula for ASC score, adding more terms in nominator and
denominator,~--- it becomes unclear how to justify it from the point of view of
coding theory. It can require modifying the definition of the channel which
we presented in that chapter.

Further, while we partially addressed the question of computing intersection
cardinalities, this result relies on knowing distributions, which renders it
unusable in practice or at least requires the use of plug-in estimators. It
would be beneficial to design a class of problems for which this issue
is eliminated. We partially did this for algorithmic problems, but at the cost
of worse performance, as noted in the discussion of that chapter.

Last but not least, there is hope that one can integrate our approach into a
toolbox of stability-related approaches (see related work), because our approach
essentially attacks the problem of identifying stability conditions (by imposing
approximations). It would be interesting to see a connection of our approach
with more conventional techniques.

\section{Robust Algorithmic Optimization}

As we have seen, the ASC score establishes a ranking of algorithms, which also
yields a corresponding ranking of their localization errors. Is this by chance or
can it be proven rigorously? Another question here arises. The
Reverse-Delete algorithm requires many more steps than Prim's and Kruskal's (see
experiments in Chapter~\ref{ch:mst}), but gains much better robustness.
Obviously, this happens because Reverse-Delete is much more elaborate in
exploring the graph: Prim's and Kruskal's algorithms eliminate unexplored edges
much more aggressively, and thus ``skip'' a lot of opportunities without
actually seeing them. We have a clear runtime vs. robustness trade-off, which
raises a question: can one construct an intermediate algorithm using the above
three as building blocks?

Another massive task would be to study more algorithmic problems from this
perspective. It must be recalled that the extension of ASC for algorithms was
performed in some sense ``blindly'', since proving the communication error
bounds (in analogy to that of Chapter~\ref{ch:gen_appch}) is hard in this case.
Consequently, it might turn out that with other algorithmic problems this
approach works much better/worse. Additional research would be beneficial here.

\section{Thermodynamic Behavior of Optimization Problems}

Although we have proven (Chapter~\ref{ch:free_energy}) the asymptotics of free
energy in two specific cases, we still did not devise a general
methodology. Although we are currently under an impression that there is no such
general methodology, more attempts should be made.

Next, we made and experimentally backed up an attractive conjecture about the
behavior of free energy in more general cases than those where
theoretic results were obtained. We brought up an intuitive explanation for it,
and it is extremely interesting to continue a line of research on that.

Further, it is still not clear how far one can go with the sparsity constraint
(which, should be recalled, was introduced to reduce the influence of
interactions between solutions without fully eliminating it). Current
condition $\log n \ll d \ll n^{2/7}$ might be extended, if one uses more
advanced bounding techniques.

Finally, but no less importantly, the free energy behavior of Gibbs-regularized
combinatorial optimization problems like sMBP looks very similar to that of REM
(Chapter~\ref{ch:smbp_and_rem}). Is this coincidental or does this tell us
anything about other aspects of their analogy? In particular, we see that they
are obviously different from the algorithmic point of view: while REM represents
total ``chaos'' with independent costs, the sMBP by definition has a certain
amount of cost dependence and thus intuitively should allow a more efficient
optimum searching than REM. Is this so, and if yes~--- can this line of
reasoning be further developed?

To understand that, one needs to properly estimate higher moments of the
log-partition function, which is highly non-trivial, but for sure a noble goal.


%!TEX root = ../gronskiy_phd_thesis.tex 
\chapter{Background}
\label{sec:background}

\hfill
\begin{minipage}[t]{.75\textwidth}
\cyrins{%
\textit{%
            ``Мы все учились понемногу \\
            Чему-нибудь и как-нибудь, \\
            Так воспитаньем, слава богу, \\
            У нас немудрено блеснуть.''}} \\[.2cm]
  \textit{(rus. 
    ``We all meandered through our schooling \\ 
     haphazard; so, to God be thanks, \\
     it's easy, without too much fooling, \\
     to pass for cultured in our ranks.'')}\\
  \hrule
  \vspace{.2cm}
  \hfill
  \textsc{--- Alexander PUSHKIN}, ``Eugene Onegin''
\end{minipage}
\\[.5cm]

In this chapter, we gather some standard background assumptions, definitions,
theorems, tools and approaches to which we refer throughout the thesis. 

\myargremark{On the notation} 
Although the thesis involves a broad range of topics and fields (coding,
combinatorics, learning, algorithms and statistical mechanics), we made an
effort to unify the notation and keep it consistent \textit{across these
fields}. While it is not always fully possible, the reader can assume that from
now on he/she will be exposed to this unified notation. This is done on purpose
to allow a simpler grasping of analogies between the same notion in different
contexts. At the cost of this, sometimes the notation might seem to be
unconventional w.r.t. each specific topic.
 
\section{Probability Theory}
It is hard to find a field nowadays where probability theory is \textit{not}
used. Being no exception, we are going to utilize the standard axiomatization of
probability theory, developed by A.N.~Kolmogorov. While we refer the reader (in
case of interest) to~\citet{shiryaev95} for details, here we list just basic
information and our assumptions to keep the notation of the thesis
self-contained.

Due to the discrete nature of the problems considered in this thesis, we will
mostly work with discrete distributions (except for data generation sources,
which will be e.g. Gaussians), hence almost all the theorems and derivations
will be stated using $\sum$ sign.

For brevity, we will omit subscripts of expectations $\Expct[\cdot]$, variances
$\Var[\cdot]$ and covariances $\Cov(\cdot, \cdot)$, except for the cases where
there are several randomness sources and we need to distinguish between them.

We will utilize the term \textit{moment-generating function}
\index{Moment-generating function} defined as follows.
\begin{definition}[Moment-generating function]
  For a random variable $X$, we call a function
  \begin{equation}
    G_X(t) \coloneqq \Expct[e^{tX}], \quad t \in \mathbb{R}
  \end{equation}
  a moment-generating function of $X$.
\end{definition}
Moment-generating functions might not exist, and in some derivations we will pay
particular attention to that.

\section{Information Theory}

\index{Information theory}
In this thesis, we are going to utilize some terminology of information theory.
While one of the best books on the topic remains to be~\citep{Cover:2006}, we
give here some basic definitions and theorems.

\subsection{Bits, Nats and the Usage of Logarithm}
\index{Nat (measure of information)}
\index{Bit (measure of information)}
It is common to use logarithms to base $2$ in information theory since they
yield an easily interpretable measure of information called \textit{bit}, which
quantifies information in \textit{binary outcomes}. 

However, one can resort, without loss of generality, to utilizing the natural
logarithm: in this case all the results of information theory hold, however, the
information measure is changed to \textit{nats}, which, informally speaking,
quantifies information in \textit{base-$e$ outcomes}.

Generally, the base-$e$ approach is characteristic for problems with a
dominant physical or general mathematical viewpoint, while the base $2$ approach
is reserved by information theory.

With a slight abuse of notation (but w.l.o.g.), we will utilize same notation
$\log(\cdot)$ for both binary and natural logarithm, and it is made clear from
the context, which of them is currently used. For a very general reference, a
binary logarithm is mainly used in Chapters~\ref{ch:gen_appch} and~\ref{ch:mst},
and a natural logarithm is used in 
Chapters~\ref{ch:free_energy} and~\ref{ch:smbp_and_rem}, since the latter ones 
are more inclined into statistical mechanics.

\subsection{Entropy and KL-Divergence}
One of the central points of this thesis is \textit{uncertainty}. This concept
obtained its rigorous definition\footnote{This very phrase ``rigorous definition
of uncertainty'' may already surprise!} through the notion of \textit{entropy}.
\index{Shannon entropy}
\index{Entropy|see{Shannon entropy}}
\begin{definition}[Shannon entropy]\label{def:shannon_entropy}
  Assume a discrete random variable $X$ is distributed according to $p(x)$. Then,
  the Shannon entropy of $X$ is defined as
  \begin{equation}
    H(X) \equiv H(p) \coloneqq -\sum_{x} p(x) \log p(x).
  \end{equation}
\end{definition}
\nomenclature[C, 00]{$H(X)$, $H(p)$}{Shannon entropy\nomnorefeqpage}%
Shannon entropy quantifies the average \textit{surprise of the outcome} $h(x)$:
\begin{equation}
  H(X) \equiv \Expct[h(x)] \equiv \Expct[ - \log p(x)].
\end{equation} 
The term \textit{surprise} refers here to self-contained information contained
in the outcome, hence the entropy refers to the average self-contained
information. The higher is the entropy, the higher is the average self-contained
information of outcomes, hence the higher is uncertainty of the distribution. A
classical example is a coin toss experiment, where the heads probability 
euqals $p$. Consider two characteristic cases: 

a) for a highly unfair coin (i.e. $p = 0$ or $p = 1$), the outcome of each 
toss does not bring much information (``surprise'') as it is known beforehand. The entropy
is the lowest (zero bits) in this case; 

b) for a fair coin (i.e. $p = 1/2$), the outcome brings a lot of information, as
we cannot have any expectations about it beforehand. The entropy is the
highest (one bit) in this case.

The joint entropy of two random variables $X$ and $Y$ is defined similarly:
\begin{equation}
  H(X, Y) \coloneqq -\sum_{x, y} p(x, y) \log p(x, y).
\end{equation}

\index{Kullback-Leibler divergence}
Next, we will use the term \textit{Kullback-Leibler divergence} (KL-divergence).
\begin{definition}[KL-divergence]
\label{def:kl_divergence}
  Assuming two discrete distributions $p(x)$ and $q(x)$, a relative entropy, or
  Kullback-Leibler (KL) divergence is defined as
  \begin{equation}
    \KL(p \| q) \coloneqq \sum_{x} p(x) \log \biggl( \frac{p(x)}{q(x)} \biggr).
  \end{equation}
  \nomenclature[C, 04]{$\KL(p \parallel q)$}{Kullback-Leibler divergence\mynomdef{def:kl_divergence}}%
  \index{Kullback-Leibler divergence}
  \index{KL|see{Kullback-Leibler divergence}}
\end{definition}
KL-divergence is one of the possible measures of ``distance'' between
distributions, and it quantifies how much additional bits (nats) of information
is required for stable communication, if the receiver assumes source
distribution $q$ instead of the true source distribution $p$.

The next important quantity is \textit{mutual information}. \index{Mutual information}
\begin{definition}[Mutual information]
\label{def:inf_theory_mutual_information}
  Assuming two random variables $X$ and $Y$, the mutual information between them
  is defined as
  \begin{equation}
    \MI(X, Y) \coloneqq \KL(p(x, y) \| p(x) p(y)) = \sum_{x,y} p(x, y)
      \log \biggl( \frac{p(x, y)}{p(x)p(y)} \biggr).
  \end{equation}
  \nomenclature[C, 04]{$\MI(X, Y)$}{mutual information\mynomdef{def:inf_theory_mutual_information}}%
  \index{MI|see{Mutual information}}
\end{definition}
Mutual information is the information-theoretic measure of independence. In other
words, one can show that $\MI(X, Y) = 0 \Leftrightarrow X \independent Y$ (symbol
``$\independent$'' denotes stochastic independence).
\nomenclature[B, 07]{$X \independent Y$}{independence of random variables}%

We will also utilize the fact that
\begin{equation}\label{eq:background_mi_decomposition}
  \MI(X, Y) = H(X) + H(Y) - H(X, Y),
\end{equation}
which can be easily observed.


\subsection{Coding Theory, Channels and Code Rates}
\label{sec:background_coding}
Another concept of information theory, whose importance is exceptionally high
for our purposes, is \textit{coding theory}. \index{Coding theory} For a
detailed explanation of the following definitions we refer the reader to~\citep[Chapter 7]{Cover:2006}, while
here we give, as usual, only necessary basics.

\index{Discrete coding channel}
\index{Channel|see{Discrete coding channel}}
\index{Memoryless channel|see{Discrete coding channel}}
\begin{definition}[Channel] \label{def:discrete_coding_channel}
  A channel is defined by an input alphabet $\mathcal{X}$, an output alphabet
  $\mathcal{Y}$, an input distribution $p(x)$ as well as a transition probability
  matrix $p(y|x)$. We denote such channel as $(\mathcal{X}, p(y|x),
  \mathcal{Y})$. A message $X$ sent through it turns into a message $Y$ received
  according to the transition matrix $p(y|x)$.
\end{definition}
\nomenclature[C, 05a]{$\mathcal{X}$, $\mathcal{Y}$}{channel input-output alphabets\mynomdef{def:discrete_coding_channel}}%
\nomenclature[C, 05b]{$(\mathcal{X}, p(y \vert x), \mathcal{Y})$}{discrete coding channel\mynomdef{def:discrete_coding_channel}}%

\index{Memoryless channel|see{Discrete coding channel}}
\begin{definition}[Memoryless channel]
  A channel is called memoryless if the probability distribution of the output 
  depends only on the input at that time and is conditionally independent of 
  previous channel inputs or outputs.
\end{definition}

\index{Code}
\index{Codebook}
\index{Codebook!Vectors}
\begin{definition}[Code] \label{def:discrete_code}
  An $(M, n)$ code for channel $(\mathcal{X}, p(y \vert x), \mathcal{Y})$ consists of
  \begin{itemize}
    \item an index set $\{1, \ldots, M\}$;
    \item an encoding rule $f^n \colon \{1, \ldots, M\} \to \mathcal{X}^n$, where 
      each $f^n(i)$ is called codebook vector. Set $\{ f^n(i)\}_{i=1}^M$ is called
      codebook.
    \item a deterministic decoding rule $g \colon \mathcal{Y}^n \to \{1, \ldots,
    M\}$.
  \end{itemize}
  \nomenclature[C, 05c]{$(M, n)$}{discrete code parameters\mynomdef{def:discrete_code}}%
\end{definition}
This definition will become highly important for understanding of Chapter~\ref{ch:gen_appch},
in conjunction with the definition of channel capacity and code rate.

\index{Channel capacity}
\begin{definition}[Channel capacity] \label{def:inf_theor_channel_capacity}
  The channel capacity of the channel $(\mathcal{X}, p(y \vert x), \mathcal{Y})$
  is defined as
  \begin{equation}
    C \coloneqq \max_{p(x)} \MI(X, Y).
  \end{equation}
  \nomenclature[C, 05d]{$C$}{channel capacity\mynomdef{def:inf_theor_channel_capacity}}%
\end{definition}

\index{Code!Rate}
\begin{definition}[Code rate] \label{def:inf_theor_code_rate}
  The rate $R_\mathrm{code}$ of an $(M, n)$ code is defined as
  \begin{equation}
    R_\mathrm{code} \coloneqq \frac{\log M}{n}
  \end{equation}
  bits (nats) per transmission.
  \nomenclature[C, 05e]{$R_\text{code}$}{code rate\mynomdef{def:inf_theor_code_rate}}%
\end{definition}

The rate is called \textit{achievable}, if there exists a sequence of
$(2^{nR_\text{code}}, n)$ codes such that the maximal probability of decoding
error goes to zero. 

Finally, it is important to realize that the capacity $C$ is the supremum of
all achievable rates, according to Shannon's Channel Coding Theorem which will be 
stated later. This fact establishes an important connection between the mutual information
and the bitrate \index{Channel!Bitrate} of the channel.

\section{Disordered Systems and Statistical Mechanics}

\index{Disordered systems}
\index{Statistical mechanics}
Large part of the thesis works with the terminology of statistical mechanics. 
A good overview is given in~\citep{bovier2012statistical}, and here we summarize
the basics specific to our work.

\subsection{Disordered Systems}
\label{sec:background_disordered_systems}

Statistical mechanics, which will play a major role in the second half of this
thesis (Chapters~\ref{ch:free_energy} and~\ref{ch:smbp_and_rem}), focuses on
systems consisting of a large number of \textit{particles}, which can find
themselves in certain aggregated \textit{states}, or \textit{configurations}.

As an example, let us consider a so-called $n$-spin system. It is represented by
some physical matter which consists of $n$ particles (atoms) $s_i$, each of
which exhibits one of the two possible directions of the magnetic moment vector
of the atom, called \textit{spin}
\begin{equation}
  \sigma_i \in \S \equiv \{-1, +1\}
\end{equation}
(i.e.~atom's magnetic field is directed ``up'' or ``down'').
\index{State|see{Statistical mechanics}}
\index{Configuration|see{Statistical mechanics}}
\index{Statistical mechanics!State}
\index{Statistical mechanics!Configuration}
\index{Spin glass}
In this system, a configuration $c_{\boldsymbol\sigma} \in
\C$ is defined as an ensemble of spins of its particles, i.e. a vector
\begin{equation}
  c_{\boldsymbol\sigma} = \langle \sigma_1, \ldots, \sigma_n \rangle. 
\end{equation}
\nomenclature[F, 20a]{$\langle \sigma_1, \ldots, \sigma_n \rangle$}{configuration of spins}%
\nomenclature[F, 20b]{$c_{\boldsymbol\sigma}$}{configuration}%
\nomenclature[F, 20c]{$J_{ij}$}{interaction matrix}%

\index{Hamiltonian}
An object used in conjunction with a configuration is its \textit{Hamiltonian},
or \textit{energy}:
\begin{equation}\label{eq:background_deterministic_hamiltonian}
  R(c_{\boldsymbol\sigma}) = -\sum_{i,j} J_{ij} \sigma_i \sigma_j,
\end{equation}
where $J_{ij}$ is a predefined interaction value between spin $i$ and spin
$j$\footnote{Other Hamiltonians are possible (e.g. the ones with external field,
etc.) which depends on the model; describing them is beyond the scope of the
thesis. A very good overview is given in~\citep{bovier2012statistical}.}.

\index{Disorder}
However, there is no randomness yet in this system, which would bring the
\textit{disorder}. To introduce it, assume that there is a random variable $X
\in \mathcal{X}$ which is distributed according to a certain law. In general case,
its nature should not necessarily be aligned with the nature of other model
elements (particles, configurations), i.e. from a model design point of view,
$X$ is a source of randomness and anything more than that. If we now assume that
the interaction matrix $J_{ij}$ depends on it, i.e. $J_{ij} = J_{ij}(X)$, the
Hamiltonian becomes random:
\begin{equation}
  R(c_{\boldsymbol\sigma}, X) = -\sum_{i,j} J_{ij}(X) \, \sigma_i \sigma_j.
\end{equation}
The described system which consists of \\
%
\hspace*{1.5\parindent} a) a source of randomness $X \in \mathcal{X}$, and \\
%
\hspace*{1.5\parindent} b) a set $\C$ of configurations\footnote{In the main part of
the thesis, we will call them ``solutions'', highlighting an optimization
prospective.} $c_{\boldsymbol\sigma}\in \C$, and \\
%
\hspace*{1.5\parindent} c) a random Hamiltonian\footnote{In the main part of the
thesis, we will call it a ``cost'', highlighting an optimization prospective.}
$R(c_{\boldsymbol\sigma}, X)$ \\ 
%
defines a \textit{disordered system}. Its nature is very closely related to the nature
of randomized optimization, and as we will see already in
Chapter~\ref{ch:gen_appch}, the ingredients of the latter are the same.

\subsection{Gibbs Measures}
The Gibbs measures are an important class of measures which arise naturally in conjunction
with statistical mechanics of disordered systems. It is defined as follows.

\begin{definition}
  The Gibbs measure over the configurations is the distribution
  \begin{equation}
    p_\beta(c | \data) =\frac{1}{Z(\beta, X)} \exp(-\beta R(c, \data)),
  \end{equation}
  where the normalization term
  \begin{equation}
    \quad Z(\beta, \data) = \sum_{c^\prime\in \C} \exp(-\beta R(c^\prime, \data))
  \end{equation}
  is called partition function.
\end{definition}

\myremark We remind the reader that the above quantities are random due to their the
parametrization by random source $X$. If it is dropped, then the Hamiltonians
turn from random into deterministic ones
(see~\eqref{eq:background_deterministic_hamiltonian} above).

\subsection{Thermodynamic Properties of Disordered Systems}
Large disordered systems are characterized by macroscopic thermodynamic
properties. One of the most important for us will be \textit{Helmholtz free
energy}. While we discuss its altered (adjusted to our needs) version in detail
later in Chapters~\ref{ch:free_energy} and~\ref{ch:smbp_and_rem}, here we state
its classical definition and provide motivation for it.

\begin{definition}\label{def:background_free_energy}
  Helmholtz free energy is originally defined as
  \begin{equation}
    \frenergy(\beta, X) = -\frac{1}{\beta}\log [Z(\beta, X)]
  \end{equation}
\end{definition}
Due to parametrization by a random source $X$, Helmholtz free energy is itself a
random variable. In this situation, studying its moments (e.g. the expected value) becomes a central
point~\citep[Chapter~9]{bovier2012statistical}. The same will hold for our case,
as explained in much more detail in Chapter~\ref{ch:free_energy} where we will define
\begin{equation}
  \frenergy(\beta) = \mathrm{const} \cdot \Expct_X \frenergy(\beta, X).
\end{equation}
The constant is not important for now.

The importance of the log-partition function is hard to overestimate: it
implicitly contains many other parameters of the system. For example, Shannon
entropy (see~Definition~\ref{def:shannon_entropy}) of the Gibbs measure can be
expressed through it:
\begin{align}
  H(p_\beta) &= -\sum_{c\in \C} p_\beta(c) \log p_\beta(c) \notag \\
    &= \log Z(\beta) - \Expct_{p_\beta(c)}[- \beta R(c)] \notag \\
    &= \log Z(\beta) - \sum_{c \in \C} \frac{- \beta R(c) e^{-\beta R(c)}}{Z(\beta)} \notag \\
    &= \log Z(\beta) - \beta \, \frac{\partial}{\partial \beta} \log Z(\beta).
\end{align}

\subsection{Maximum Entropy Principle}
\label{sec:background_max_entropy}

\index{Maximum entropy principle}
\index{Testable information}
\index{Occam's razor}
In the following, we will refer to Jaynes' \textit{maximum entropy}
principle~\citep{Jaynes57a,Jaynes57b}. This principle states, informally
speaking, that the current state of a physical system is represented by a
distribution which has the highest possible entropy under the given constraints.
This concept is related to a commonly used Occam's razor principle which states
that everything should be explained in the simplest possible way given the
observed constraints.

More formally speaking, under the observed data $X$, the distribution which is
followed by a physical system is the following:
\begin{equation}\label{eq:background_max_ent}
  \hat p(c) = \arg \max_{p \colon \mathrm{obs}(p) = X} H(p),
\end{equation}
where $\mathrm{obs}(p)$ is the function which yields observed data (also-called
\textit{testable information}; for example, it extracts moments of 
the distribution~--- and thus in some sense ``tests'' it).

We will largely utilize the fact that the maximum entropy distribution under certain
constraints is the Gibbs distribution:
\begin{align}
  p_\beta(c|\data) &=\frac{1}{Z(\beta, X)} \exp(-\beta R(c, \data))
    \quad\text{with} \notag \\
  \quad Z(\beta, \data) &= \sum_{c^\prime\in \C} \exp(-\beta R(c^\prime, \data))\,,
\end{align}
where $R(c, X)$ is determined explicitly by the context, and $\beta$ comes from
the method of Lagrange multipliers when optimizing~\eqref{eq:background_max_ent}
and has a semantics of an \textit{inverse temperature}.



%!TEX root = ../gronskiy_phd_thesis.tex
\chapter*{Acknowledgments}

First of all, I am deeply grateful to my advisor, Prof. Dr. Joachim M. Buhmann,
who gave me an opportunity to be part of his group and who, although allowing me
much freedom in choosing topics, always guaranteed me necessary help and
provided guidance. Besides infallible scientific interest, he always supported
me in non-professional matters.

I would also like to thank my co-examiners, Prof. Dr. Peter Widmayer from the
ETH Zurich and Prof. Dr. Wojciech Szpankowski from Purdue University for taking
time to read through this opus, give valuable comments and examine me.
Throughout my studies, Peter Widmayer kept encouraging me and was involved in
numerous discussions on a regular basis. Wojciech Szpankowski became my major
collaborator for a large part of the time I spent on my PhD. He shared valuable
ideas, supported me, taught me a word ``prodding'' and showed what it meant in
practice. He also organized my and my student's visit to Purdue, during which we
made significant progress.

I will always be grateful to my first university lecturer and scientific advisor
in Moscow, Prof. Dr. Alexander Ugolnikov, who is, unfortunately, no longer with
us. He introduced me to the field of discrete mathematics and inspired me to
combine humor and seriousness, not forgetting about life while doing research.

Apart from mentioned above, thanks to my close collaborators with whom I had
lots of fruitful discussions on the topics of this thesis: Tobias Pr\"oger (who
also kindly agreed to proofread this thesis), Rastislav \v{S}r\'amek, Paolo
Penna, Mat{\'{u}}s Mihal{\'{a}}k, An Bian, Nico Gorbach and Stefan Bauer.

The Information Science and Engineering group (formerly the Machine Learning
group) at the ETH Zurich accompanied me from the very first days in Zurich, and I
enjoyed being part of it. Colleagues from my group, as well as from the ones of
Prof. Dr. Andreas Krause (LAS group) and Prof. Dr. Thomas Hofmann (DALab),
taught me a lot during the numerous occasions we had to interact. 
Thanks to David, Brian, Sharon, Patrick, Gabriel, Hasta, Morteza, Alberto,
Ludwig, Peter, Dwarikanath, Dima, Kate, Judith, Rebekka, Luis, Nico, Stefan,
Djordje, An, Luca, Viktor, Alina and Aytun\c{c} for a wonderful time.

If one sees a scientific group as an organism, then heads and bodies have
already been mentioned. But a group is still nothing without its heart which
actually pumps life into it: thanks to our great administrators Rita Klute and
Marianna Berger for their friendliness, support, help and organizing skills.

I thank my Master's students Julien, George and Edouard, with whom we worked a
lot and learned much together. I also constantly learned from numerous
students of the ETH who attended our Machine Learning and Statistical Learning
Theory courses.

I was lucky to spend a summer with Google Research in Zurich, and I am
grateful to my host Neil Houlsby, who was supporting and patient towards me as I
slowly progressed through my internship, and to the rest of the hosting group:
Massimiliano, Jannis, Christian and Wojciech.

The life of a PhD student is, of course, not all about studying and research,
but also about \textit{simply} life~--- in all its variety. Thanks to my Russian
friends Dima, Valya, Valera, Masha, Kate, Roma, Martin, Nikolay, Alexander,
Arseniy, Anna, Vita and Iuliya for their support, good mood and for the great
time we spent already and still spend together. Further, while doing research,
one must periodically clear head with hobbies, so thanks to SWISS Flying Club,
whose instructors introduced me to the art of flying an airplane and helped to
see Switzerland from a bird's-eye view; thanks to the Zurich Rescue and
Ambulance Service, whose instructors accepted me as a volunteer and persevered
to train me, despite my awful Swiss German in the very beginning. Thanks also to
ASVZ and Kim Dojo Karate Clubs whose members actively helped me not to become
overly self-confident at times of my research euphoria.

Thanks to my loving parents Yury and Natalya, who raised me not mechanistically,
but rather showed by their own example what it means~--- to be genuinely
interested in life. To always be curious, open, to keep pushing and not give up.
They are both mathematicians, and I acquired a large part of my early interest
in programming and mathematics from them. My younger brothers Dima and Ivan are
my first and best friends, and we keep learning from each other till today.

And most importantly, infinite (as in ``$\infty$'') gratitude goes to my loving
wife Lena. The level of encouragement and support I get from her is more than
enormous. She laughs and makes me laugh even when everything seems ruined and
frustrating. She is supporting in whatever we do together. Holding PhD herself,
she understands me as nobody else ever did. Thank you, my darling, for being
there. The fact that this text is eventually out is your achievement no less
than mine. 

\textit{UPD on the 10$\,^{th}$ of March 2018.} Four days after the defense of this
thesis, my wife Lena gave birth to our wonderful son Andrey. Existence of such
miracles as an emerging human life is the greatest support, motivation, and
reward for me.

\newpage

\hbox{}

\vfil

\begin{flushright}
\noindent 
\textit{To our parents, who gave us the best they had.} \\[.15cm]
\textit{To my wife, with whom we try to transform it into something even
better.} \\[.15cm]
\textit{To our coming children, who will in turn receive it from us \\ and
hopefully take over this ever-recurring task.}
\end{flushright}

\vfil\vfil\vfil\vfil\vfil

\hbox{}
